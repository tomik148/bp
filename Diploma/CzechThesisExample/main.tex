% Nejprve uvedeme tridu dokumentu s volbami
\documentclass[czech,bachelor,dept460,male,csharp]{diploma}
% Dalsi doplnujici baliky maker
\usepackage[autostyle=true,czech=quotes]{csquotes} % korektni sazba uvozovek, podpora pro balik biblatex
\usepackage[backend=biber, style=iso-numeric, alldates=iso]{biblatex} % bibliografie
\usepackage{dcolumn} % sloupce tabulky s ciselnymi hodnotami
\usepackage{subfig} % makra pro "podobrazky" a "podtabulky"

\newcommand{\AveSoft}{AVE~Soft~s.r.o.}
\newcommand{\EvolioEight}{Evolio~8}
\newcommand{\EFilters}{Evolio~Power~Reporting}
\newcommand{\EvolioX}{Evolio}
\newcommand{\EData}{EData}
\newcommand{\Csharp}{%
  {\settoheight{\dimen0}{C}C\kern-.05em \resizebox{!}{\dimen0}{\raisebox{\depth}{\#}}}}

% Zadame pozadovane vstupy pro generovani titulnich stran.
\ThesisAuthor{Tomáš Chalupa}

\CzechThesisTitle{Absolvování individuální odborné praxe}

\EnglishThesisTitle{Individual Professional Practice in the Company}

\SubmissionDate{30. dubna 2020}

% Pokud nechceme nikomu dekovat makro zapoznamkujeme.
\Thanks{Rád bych na tomto místě poděkoval všem, kteří mi s prací pomohli, protože bez nich by tato práce nevznikla.}

% Zadame cestu a jmeno souboru ci nekolika souboru s digitalizovanou podobou zadani prace.
% Pokud toto makro zapoznamkujeme sazi se stranka s upozornenim.
\ThesisAssignmentImagePath{Figures/Assignment}

% Zadame soubor s digitalizovanou podobou prohlaseni autora zaverecne prace.
% Pokud toto makro zapoznamkujeme sazi se cisty text prohlaseni.
%\AuthorDeclarationImageFile{Figures/AuthorDeclaration.jpg}


% Zadame soubor s digitalizovanou podobou souhlasu spolupracujici prav. nebo fyz. osoby.
% Pokud toto makro zapoznamkujeme sazi se cisty text souhlasu.
\CooperatingPersonsDeclarationImageFile{Figures/CoopPersonDeclaration.jpg}

\CzechAbstract{Tato práce popisuje průběh mé bakalářské práce formou individuální odborné praxe ve firmě \AveSoft. Ve firmě jsem působil jako \Csharp\ .NET DEVELOPER. Mým úkolem bylo vytvořit pomocnou aplikaci, k již existujícímu informačnímu systému Evolio, která by zvládla pracovat s velkým množstvím dat.}

\CzechKeywords{Bakalářská praxe; Informační systém; \Csharp; WPF}

\EnglishAbstract{This paper discribes my bachelor's thesis by form of individual apprenticeship in company \AveSoft.}

\EnglishKeywords{ Bachelor thesis; Information system; \Csharp; WPF}

\AddAcronym{WPF}{Windows Presentation Foundation}
\AddAcronym{HTML}{Hyper Text Markup Language}
\AddAcronym{VB.NET}{Visual Basic .NET}
\AddAcronym{ASP}{Active Server Pages}
\AddAcronym{API}{Application Programming Interface}
\AddAcronym{REST}{Representational State Transfer}
\AddAcronym{SQL}{Structured Query Language}

%\addbibresource{biblatex-examples.bib}
%\addbibresource{coffee.bib}

% Novy druh tabulkoveho sloupce, ve kterem jsou cisla zarovnana podle desetinne carky
\newcolumntype{d}[1]{D{,}{,}{#1}}


% Zacatek dokumentu
\begin{document}

% Nechame vysazet titulni strany.
\MakeTitlePages

% A nasleduje text zaverecne prace.
\section{Úvod}
TODO!!

\section{O Firmě}
	\AveSoft\ je softwarová společnost založena roku 1997 vyvíjející informační systémy pro exekutory a právní kanceláře, také vytváří software na míru.
	Mezi jejich zákazníky patří české pobočky prodejců automobilů Opel, životní pojišťovna Wustenrot, D.A.S. - Pojišťovna pro právní výdaje.
	\AveSoft\ získala řadu ocenění, včetně druhého místa v soutěži české mobilní aplikace 2012 nebo finalisty Microsoft Industry Awards 2007 a 2012.
	\subsection{Starupy}
	Vedle klasický produktů u nás vznikly projekty, které svým přesahem a zaměřením vyžadovaly odlišný přístup. Vzdali jsme se proto pohodlí běžné IT společnosti a projekty osamostatnili jako plnohodnotné startupy. Rozhodně nelitujeme. 
 		\subsubsection{EXDRAZBY.CZ}
 		V roce 2010 byl změněn zákon, který umožnil provádět dražby nemovitostí online. Byla to skvělá příležitost pro rozjezd nového projektu exdrazby.cz.

		Po osmi měsících vývoje jsme na trh uvedli ostrou verzi dražebního portálu a už v prvním roce se stal největším na trhu. V dalších letech svou pozici ještě posílil. Klíčovým faktorem úspěchu byla sada nadstandardních služeb rozšiřující základní službu provedení aukce.

		Za prvních pět let provozu byly prostřednictvím exdrazby.cz vydraženy nemovitosti v hodnotě více než 8 miliard korun.
 		\subsubsection{PRESENTIGO}
 		S Presentigem jsme zažili opravdový global startup se vším všudy. Naší hlavní orientací je trh USA a proto jsme v  Silicon Valey strávili necelý rok. Dnes máme zákazníky převážně v USA a ČR. Získali jsme investici 500 tisíc dolarů a řadu zajímavých zákazníků, např. E.ON, O2, UPC, Škoda Auto atd.

		Presentigo pomáhá obchodním týmům více prodávat. Firmám přináší dva základní benefity:
		\begin{itemize}
			\item
			Digitalizace obchodního procesu. Presentigo sbírá data z terénu, které pak využívají markeťáci i obchodní manažeři. Kromě toho zjednodušujeme práci obchodníkům.
			\item
			Na klíč vytváříme prezentace, které zákazníky zaujmou a zároveň pomohou dobře vysvětlit přínosy nabízeného produktu nebo služby.
		\end{itemize}

 		\subsubsection{MIXIEW}
 		Myšlenka vytvořit Mixiew vzešla z jednoduché úvahy: Každý amatérský závodník chce mít ze závodu nějaké fotky a video. Co takhle posbírat záznamy od profesionálních kameramanů i diváků na tratit a sestříhat originální video pro každého závodníka?

		Aby celá myšlenka dávala smysl, bylo nutné najít funkční business model. Ukázalo se, že do sportu teče obrovské množství peněz od sponzorů. Ti stále hledají nové způsoby, jak tyto peníze vynaložit efektivně a měřitelně. Rozdávání reklamníchí klíčenek je rozhodně za zenitem.

		Závodníci svá videa z Mixiew sdílí na sociálních sítích a tak sponzor získá možnost propagace své značky nejen vůči závodníkům, ale i v okruhu jejich kontaktů.
	\subsection{Firemní Software}
		\subsubsection{\EvolioEight}
		{\EvolioEight} je desktopová aplikace napsaná v {\Csharp} WPF s jádrem z předchozí verze napsané v VB.NET. Je to informační systém navržený pro použití v advokacii a příbuzných oborech. 
		\subsubsection{\EvolioX}
		{\EvolioX} je webová aplikace napsaná v TypeScriptu a backend v {\Csharp} ASP.NET Core MVC. Je přímím nástupcem {\EvolioEight}.
		\subsubsection{\EData}
		{\EData} je webové REST API napsané v {\Csharp} ASP.NET Core MVC.
		Slouží především k přístupu do databázi pro {\EvolioX} a {\EFilters}.
\section{Moje úkoly}
	Moje hlavní náplní praxe byl vývoj aplikace {\EFilters} a také jsem řešil drobnosti co se rozbily v {\EvolioEight}.
	\subsection{\EFilters}
		{\EFilters} je desktopová aplikace kterou jsem mně za úkol vytvořit, je to je to ta část aplikace {\EvolioEight} která by jako webová aplikace fungovala hůře.
		Ta část která se v {\EvolioEight} nazývala "Filtry".
		Filtry jsou uživatelem napsané SQL dotazy, pro optimalizaci exekučního plánu jsou uloženy jako procedury, nad kterýma lze provádět hromadné operace, ať už přes SQL nebo jiné interní systémy.
		Důvodem proč pro toto webová aplikace není vhodná je že nějaké filtry mohou vrátit veliké množství dat, až v řádech milionů záznamů.
		\subsubsection{Demo aplikace}
			Prvním krokem při vývoji této aplikace bylo zjistit jestli lze vytvořit dostatečně rychlou aplikaci která by zvládla zhruba milion záznamů a začala je zobrazovat ihned po tom co je dostala ze serveru namísto toho že by si je nejprve všechny stáhla a až pak je zobrazila jak tomu bylo u předchozí aplikace {\EvolioEight}.
			Taktéž tato demo aplikace mi sloužila jako seznámení s WPF se kterými jsem dosud nepracoval.
		\subsubsection{Serializace}

		\subsubsection{WPF}

		\subsubsection{Editor Filtrů}
\section{Závěr}


\printbibliography[title={Literatura}, heading=bibintoc]


%\appendix
%\section{Plné tkví drah pokles průběhu}
%TODO!!

\end{document}
